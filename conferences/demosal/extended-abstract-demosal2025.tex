% Options for packages loaded elsewhere
\PassOptionsToPackage{unicode}{hyperref}
\PassOptionsToPackage{hyphens}{url}
\PassOptionsToPackage{dvipsnames,svgnames,x11names}{xcolor}
%
\documentclass[
  12pt,
  letterpaper,
]{article}

\usepackage{amsmath,amssymb}
\usepackage{setspace}
\usepackage{iftex}
\ifPDFTeX
  \usepackage[T1]{fontenc}
  \usepackage[utf8]{inputenc}
  \usepackage{textcomp} % provide euro and other symbols
\else % if luatex or xetex
  \usepackage{unicode-math}
  \defaultfontfeatures{Scale=MatchLowercase}
  \defaultfontfeatures[\rmfamily]{Ligatures=TeX,Scale=1}
\fi
\usepackage{lmodern}
\ifPDFTeX\else  
    % xetex/luatex font selection
\fi
% Use upquote if available, for straight quotes in verbatim environments
\IfFileExists{upquote.sty}{\usepackage{upquote}}{}
\IfFileExists{microtype.sty}{% use microtype if available
  \usepackage[]{microtype}
  \UseMicrotypeSet[protrusion]{basicmath} % disable protrusion for tt fonts
}{}
\makeatletter
\@ifundefined{KOMAClassName}{% if non-KOMA class
  \IfFileExists{parskip.sty}{%
    \usepackage{parskip}
  }{% else
    \setlength{\parindent}{0pt}
    \setlength{\parskip}{6pt plus 2pt minus 1pt}}
}{% if KOMA class
  \KOMAoptions{parskip=half}}
\makeatother
\usepackage{xcolor}
\usepackage[left=2.54cm,right=2.54cm,top=2.54cm,bottom=2.54cm]{geometry}
\setlength{\emergencystretch}{3em} % prevent overfull lines
\setcounter{secnumdepth}{-\maxdimen} % remove section numbering


\providecommand{\tightlist}{%
  \setlength{\itemsep}{0pt}\setlength{\parskip}{0pt}}\usepackage{longtable,booktabs,array}
\usepackage{calc} % for calculating minipage widths
% Correct order of tables after \paragraph or \subparagraph
\usepackage{etoolbox}
\makeatletter
\patchcmd\longtable{\par}{\if@noskipsec\mbox{}\fi\par}{}{}
\makeatother
% Allow footnotes in longtable head/foot
\IfFileExists{footnotehyper.sty}{\usepackage{footnotehyper}}{\usepackage{footnote}}
\makesavenoteenv{longtable}
\usepackage{graphicx}
\makeatletter
\def\maxwidth{\ifdim\Gin@nat@width>\linewidth\linewidth\else\Gin@nat@width\fi}
\def\maxheight{\ifdim\Gin@nat@height>\textheight\textheight\else\Gin@nat@height\fi}
\makeatother
% Scale images if necessary, so that they will not overflow the page
% margins by default, and it is still possible to overwrite the defaults
% using explicit options in \includegraphics[width, height, ...]{}
\setkeys{Gin}{width=\maxwidth,height=\maxheight,keepaspectratio}
% Set default figure placement to htbp
\makeatletter
\def\fps@figure{htbp}
\makeatother
% definitions for citeproc citations
\NewDocumentCommand\citeproctext{}{}
\NewDocumentCommand\citeproc{mm}{%
  \begingroup\def\citeproctext{#2}\cite{#1}\endgroup}
\makeatletter
 % allow citations to break across lines
 \let\@cite@ofmt\@firstofone
 % avoid brackets around text for \cite:
 \def\@biblabel#1{}
 \def\@cite#1#2{{#1\if@tempswa , #2\fi}}
\makeatother
\newlength{\cslhangindent}
\setlength{\cslhangindent}{1.5em}
\newlength{\csllabelwidth}
\setlength{\csllabelwidth}{3em}
\newenvironment{CSLReferences}[2] % #1 hanging-indent, #2 entry-spacing
 {\begin{list}{}{%
  \setlength{\itemindent}{0pt}
  \setlength{\leftmargin}{0pt}
  \setlength{\parsep}{0pt}
  % turn on hanging indent if param 1 is 1
  \ifodd #1
   \setlength{\leftmargin}{\cslhangindent}
   \setlength{\itemindent}{-1\cslhangindent}
  \fi
  % set entry spacing
  \setlength{\itemsep}{#2\baselineskip}}}
 {\end{list}}
\usepackage{calc}
\newcommand{\CSLBlock}[1]{\hfill\break\parbox[t]{\linewidth}{\strut\ignorespaces#1\strut}}
\newcommand{\CSLLeftMargin}[1]{\parbox[t]{\csllabelwidth}{\strut#1\strut}}
\newcommand{\CSLRightInline}[1]{\parbox[t]{\linewidth - \csllabelwidth}{\strut#1\strut}}
\newcommand{\CSLIndent}[1]{\hspace{\cslhangindent}#1}

% -----------------------
% CUSTOM PREAMBLE STUFF
% -----------------------

% -----------------
% Typography tweaks
% -----------------
% Indent size
\setlength{\parindent}{1pc}  % 1p0

% Fix widows and orphans
\usepackage[all,defaultlines=2]{nowidow}

% List things
\usepackage{enumitem}
% Same document-level indentation for ordered and ordered lists
\setlist[1]{labelindent=\parindent}
\setlist[itemize]{leftmargin=*}
\setlist[enumerate]{leftmargin=*}

% Wrap definition list terms
% https://tex.stackexchange.com/a/9763/11851
\setlist[description]{style=unboxed}


% For better TOCs
\usepackage{tocloft}

% Remove left margin in lists inside longtables
% https://tex.stackexchange.com/a/378190/11851
\AtBeginEnvironment{longtable}{\setlist[itemize]{nosep, wide=0pt, leftmargin=*, before=\vspace*{-\baselineskip}, after=\vspace*{-\baselineskip}}}

% Allow for /singlespacing and /doublespacing
\usepackage{setspace}


% -----------------
% Title block stuff
% -----------------

% Abstract
\usepackage[overload]{textcase}
\usepackage[runin]{abstract}
\renewcommand{\abstractnamefont}{\sffamily\footnotesize\bfseries\MakeUppercase}
\renewcommand{\abstracttextfont}{\sffamily\small}
\setlength{\absleftindent}{\parindent * 2}
\setlength{\absrightindent}{\parindent * 2}
\abslabeldelim{\quad}
\setlength{\abstitleskip}{-\parindent}


% Keywords
\newenvironment{keywords}
{\vskip -3em \hspace{\parindent}\small\sffamily{\sffamily\footnotesize\bfseries\MakeUppercase{Keywords}}\quad}
{\vskip 3em}

  
% Title
\usepackage{titling}
\setlength{\droptitle}{3em}
\pretitle{\par\vskip 5em \begin{flushleft}\LARGE\sffamily\bfseries}
\posttitle{\par\end{flushleft}\vskip 0.75em}


% Authors
%
% PHEW this is complicated for a number of reasons!
%
% When using \and with multiple authors, the article class in LaTeX wraps each 
% author block in a tabluar environment with a hardcoded center alignment. It's 
% possible to use \preauthor{} to start tabulars with a left alignment {l}, but 
% that only applies to the first author because the others all use \and with the 
% hardcoded {c}. But we can override the \and command and add our own {l}
%
% (See https://github.com/rstudio/rmarkdown/issues/1716#issuecomment-560601691 
% for an example of redefining \and to just be \\)
%
% That's all great, except tabulars have some amount of default horizontal 
% padding, which makes left-aligned author blocks not actuall get fully 
% left-aligned on the page. We can set the horizontal padding for the column to 
% 0, but it requires some wonky syntax: {@{\hspace{0em}}l@{}}
\renewcommand{\and}{\end{tabular} \hskip 3em \begin{tabular}[t]{@{\hspace{0em}}l@{}}}
\preauthor{\begin{flushleft}
           \lineskip 1.5em 
           \begin{tabular}[t]{@{\hspace{0em}}l@{}}}
\postauthor{\end{tabular}\par\end{flushleft}}

% Omit the date since the \published command does that
\predate{}
\postdate{}

% Command for a note at the top of the first page describing the publication
% status of the paper.
\newcommand{\published}[1]{%
   \gdef\puB{#1}}
   \newcommand{\puB}{}
   \renewcommand{\maketitlehooka}{%
       \par\noindent\footnotesize\sffamily \puB}


% ------------------
% Section headings
% ------------------
\usepackage{titlesec}
\titleformat*{\section}{\Large\sffamily\bfseries\raggedright}
\titleformat*{\subsection}{\large\sffamily\bfseries\raggedright}
\titleformat*{\subsubsection}{\normalsize\sffamily\bfseries\raggedright}
\titleformat*{\paragraph}{\small\sffamily\bfseries\raggedright}

% \titlespacing{<command>}{<left>}{<before-sep>}{<after-sep>}
% Starred version removes indentation in following paragraph
\titlespacing*{\section}{0em}{2em}{0.1em}
\titlespacing*{\subsection}{0em}{1.25em}{0.1em}
\titlespacing*{\subsubsection}{0em}{0.75em}{0em}


% -----------
% Footnotes
% -----------
% NB: footmisc has to come after setspace and biblatex because of conflicts
\usepackage[bottom, flushmargin]{footmisc}
\renewcommand*{\footnotelayout}{\footnotesize}

\addtolength{\skip\footins}{10pt}    % vertical space between rule and main text
\setlength{\footnotesep}{5pt}  % vertical space between footnotes


% ----------
% Captions
% ----------
\usepackage[font={small,sf}, labelfont={small,sf,bf}]{caption}


% --------
% Macros
% --------
% pandoc will not convert text within \begin{} XXX \end{} to Markdown and will
% treat it as regular TeX. Because of this, it's impossible to do stuff like
% this:

% \begin{landscape}
%
% | One | Two   |
% |-----+-------|
% | my  | table |
% | is  | nice  |
%
% \end{landscape}
%
% Since it'll render like: | One | Two | |—–+——-| | my | table | | is | nice |
% 
% BUT, from this http://stackoverflow.com/a/41945462/120898 we can get around
% this by creating new commands for \begin and \end, like this:
\usepackage{pdflscape}
\newcommand{\blandscape}{\begin{landscape}}
\newcommand{\elandscape}{\end{landscape}}

% \blandscape
%
% | One | Two   |
% |-----+-------|
% | my  | table |
% | is  | nice  |
%
% \elandscape

% Same thing, but for generic groups
% But can't use \bgroup and \egroup because those are built-in TeX things
\newcommand{\stgroup}{\begingroup}
\newcommand{\fingroup}{\endgroup}


% ---------------------------
% END CUSTOM PREAMBLE STUFF
% ---------------------------
\makeatletter
\@ifpackageloaded{caption}{}{\usepackage{caption}}
\AtBeginDocument{%
\ifdefined\contentsname
  \renewcommand*\contentsname{Table of contents}
\else
  \newcommand\contentsname{Table of contents}
\fi
\ifdefined\listfigurename
  \renewcommand*\listfigurename{List of Figures}
\else
  \newcommand\listfigurename{List of Figures}
\fi
\ifdefined\listtablename
  \renewcommand*\listtablename{List of Tables}
\else
  \newcommand\listtablename{List of Tables}
\fi
\ifdefined\figurename
  \renewcommand*\figurename{Figure}
\else
  \newcommand\figurename{Figure}
\fi
\ifdefined\tablename
  \renewcommand*\tablename{Table}
\else
  \newcommand\tablename{Table}
\fi
}
\@ifpackageloaded{float}{}{\usepackage{float}}
\floatstyle{ruled}
\@ifundefined{c@chapter}{\newfloat{codelisting}{h}{lop}}{\newfloat{codelisting}{h}{lop}[chapter]}
\floatname{codelisting}{Listing}
\newcommand*\listoflistings{\listof{codelisting}{List of Listings}}
\makeatother
\makeatletter
\makeatother
\makeatletter
\@ifpackageloaded{caption}{}{\usepackage{caption}}
\@ifpackageloaded{subcaption}{}{\usepackage{subcaption}}
\makeatother
\ifLuaTeX
  \usepackage{selnolig}  % disable illegal ligatures
\fi
\usepackage{bookmark}

\IfFileExists{xurl.sty}{\usepackage{xurl}}{} % add URL line breaks if available
\urlstyle{same} % disable monospaced font for URLs
\hypersetup{
  pdftitle={Medición de percepciones y preferencias sobre meritocracia en etapa escolar en Chile},
  pdfauthor={Juan Carlos Castillo; Andreas Laffert},
  colorlinks=true,
  linkcolor={DarkSlateBlue},
  filecolor={Maroon},
  citecolor={DarkSlateBlue},
  urlcolor={DarkSlateBlue},
  pdfcreator={LaTeX via pandoc}}

% -----------------------
% END-OF-PREAMBLE STUFF
% -----------------------



% ---------------------- 
% Title block elements
% ---------------------- 
\usepackage{orcidlink}  % Create automatic ORCID icons/links

\title{Medición de percepciones y preferencias sobre meritocracia en
etapa escolar en Chile}


\author{
{\large Juan Carlos Castillo~\orcidlink{0000-0003-1265-7854}}%
 \\%
Departmento de Sociología, Universidad de Chile \\%
{\footnotesize \url{juancastillov@uchile.cl}} \and
{\large Andreas Laffert~\orcidlink{0000-0002-9008-2454}}%
 \\%
Instituto de Sociología, Universidad Católica de Chile \\%
{\footnotesize \url{alaffertt@estudiante.uc.cl}} \and
}

\date{}


% Typeset URLs in the same font as their parent environment
%
% This has to come at the end of the preamble, after any biblatex stuff because 
% some biblatex styles (like APA) define their own \urlstyle{}
\usepackage{url}
\urlstyle{same}

% ---------------------------
% END END-OF-PREAMBLE STUFF
% ---------------------------
\begin{document}
% ---------------
% TITLE SECTION
% ---------------
\published{\textbf{}}

\maketitle



% -------------------
% END TITLE SECTION
% -------------------


\setstretch{1.15}
\section{Introducción}\label{introducciuxf3n}

La creencia en que las disparidades económicas se justifican por
diferencias en elementos meritocráticos --- como esfuerzo y talento
individual (\citeproc{ref-young_rise_1958}{Young, 1958}) --- han sido
identificadas como mecanismos clave para explicar la persistencia de las
desigualdades. Desde sus comienzos, las instituciones educativas han
sido fundamentales en la promoción de este tipo de valores, dado su
vínculo con las promesas de movilidad social y mejores oportunidades de
vida. Ante los desafíos derivados de las diferentes conceptualizaciones
y mediciones de la meritocracia, Castillo et al.
(\citeproc{ref-castillo_multidimensional_2023}{2023}) proponen un marco
conceptual y de medición para evaluar las percepciones y preferencias
meritocráticas y no meritocráticas. Con el propósito de contribuir a la
investigación empírica sobre la formación de la meritocracia y sus
factores relacionados en edades tempranas
(\citeproc{ref-batruch_belief_2022}{Batruch et al., 2022};
\citeproc{ref-darnon_where_2018}{Darnon et al., 2018};
\citeproc{ref-wiederkehr_belief_2015}{Wiederkehr et al., 2015}), este
estudio se propone evaluar la aplicabilidad de esta escala en población
escolar en Chile, un país caracterizado por una aguda y persistente
desigualdad económica y que cuenta un sistema educacional altamente
estratificado (\citeproc{ref-chancel_world_2022}{Chancel et al., 2022};
\citeproc{ref-corvalan_mercado_2017}{Corvalán et al., 2017}).

Argumentamos en primer lugar que existe una distinción entre percepción
y preferencias en la meritocracia. Mientras la percepción se asocia a
como las personas observan el funcionamiento de principios
meritocráticos en la sociedad (lo que es), las preferencias se refieren
a juicios normativos (lo que debería ser). La segunda distinción tiene
que ver con elementos meritocráticos y no meritocráticos. En este caso,
se considera también que aspectos como el rol de los contactos
personales y la riqueza familiar no son necesariamente opuestos a la
percepción y valoración del esfuerzo y del talento en la obtención de
logros y recompensas.

Para poder establecer en qué medida es posible reconocer las distintas
dimensiones de la meritocracia en la población escolar se implementará
un procedimiento de análisis factorial confirmatorio en datos de
estudiantes chilenos. Además, se realizarán estimaciones de invarianza
longitudinal entre dos olas de una muestra de estudiantes en Chile.
Finalmente, planteamos que los factores meritocráticos se asocian
positivamente con las preferencias por justicia de mercado en salud,
pensiones y educación (\citeproc{ref-lindh_public_2015}{Lindh, 2015}),
mientras lo opuesto ocurre para los factores no meritocráticos.

\section{Hipótesis}\label{hipuxf3tesis}

\begin{itemize}
\tightlist
\item
  \(H_1\): La percepción de meritocracia es una variable latente
  construida a partir de indicadores que miden la importancia atribuida
  al talento y al esfuerzo para lograr mayores recompensas en la vida.
\item
  \(H_2\): La percepción no meritocrática es una variable latente
  derivada de indicadores que reflejan el acuerdo con afirmaciones de
  que las personas con contactos personales y padres adinerados tienen
  más probabilidades de tener éxito en la vida.
\item
  \(H_3\): Las preferencias meritocráticas funcionan como una variable
  latente basada en el valor normativo asignado al esfuerzo y al
  talento.
\item
  \(H_4\): Las preferencias no meritocráticas funcionan como una
  variable latente basada en el valor normativo asignado al uso de
  contactos personales y tener padres adinerados.
\item
  \(H_5\): Las escalas de meritocracia y no meritocracia presentan
  invarianza métrica entre las dos olas del estudio.
\item
  \(H_6\): Los factores meritocráticos están positivamente asociados con
  las preferencias por la justicia de mercado en salud, pensiones y
  educación, mientras que lo contrario se observa para los factores no
  meritocráticos.
\end{itemize}

\section{Datos, variables y métodos}\label{datos-variables-y-muxe9todos}

\subsection{Datos}\label{datos}

Este estudio se basa en la información proporcionada por la base de
datos de la Encuesta Panel Educación y Meritocracia (EDUMER) en sus olas
de 2023 (N = 839) y 2024 (N = 612) para estudiantes. Esta base de datos
se sustenta en la aplicación de cuestionarios web a estudiantes de sexto
básico y primero medio provenientes de 9 escuelas de la región
Metropolitana y Valparaíso de Chile.

\subsection{Variables}\label{variables}

\textbf{Escala de percepciones y preferencias sobre meritocracia}: Las
variables incluidas en el modelo de medición sobre percepciones y
preferencias de meritocracia y no meritocracia se operacionalizan según
los ítems propuestos por Castillo et al.
(\citeproc{ref-castillo_multidimensional_2023}{2023}). La percepción de
la meritocracia se mide mediante dos ítems que indagan sobre el grado de
acuerdo con que el esfuerzo y la habilidad son recompensados en Chile,
mientras que la percepción no meritocrática se mide con dos ítems que
evalúan el grado de acuerdo sobre el éxito asociado a contactos y
riqueza familiar. La preferencia por la meritocracia se mide con dos
ítems que evalúan el acuerdo en que quienes más se esfuerzan o tienen
más talento deberían ser más recompensados. La preferencia por aspectos
no meritocráticos se mide con dos indicadores que evalúan el acuerdo en
que está bien que quienes tienen mejores contactos o padres ricos tengan
más éxito. Cada ítem se contestó en una escala Likert de cuatro puntos
que va desde ``muy en desacuerdo'' (1) hasta ``muy de acuerdo'' (4).

\textbf{Preferencias por justicia de mercado}: Este constructo se mide a
través de tres variables que abordan el grado de justificación respecto
a si el acceso a los servicios sociales en salud, pensiones y educación
debe estar condicionado por los ingresos. La justificación de la
desigualdad en salud se mide a través del ítem: «¿Está bien que aquellos
que puedan pagar más tengan mejor acceso a salud?». La misma pregunta se
hace para pensiones y educación. En todos los casos, los encuestados
indican sus preferencias en una escala Likert que va desde ``muy en
desacuerdo'' (1) hasta ``muy de acuerdo'' (4). Además, incluimos un
indicador resumido de ``preferencias por la justicia de mercado'',
medido por un índice promedio de todos estos ítems (\(\alpha\) = 0,83),
con valores que van de 1 a 4, donde los valores más altos representan
mayores preferencias por justicia de mercado.

\subsection{Métodos}\label{muxe9todos}

Para evaluar nuestras hipótesis, empleamos análisis factoriales
confirmatorios (CFA) basado en un modelo de medición de cuatro factores
latentes (\citeproc{ref-castillo_multidimensional_2023}{Castillo et al.,
2023}) con estimador Diagonal Weighted Least Squares (DWLS) debido al
nivel ordinal de los ítems (\citeproc{ref-kline_principles_2023}{Kline,
2023}). La evaluación del ajuste del modelo se fundamentó en los
criterios propuestos por Brown
(\citeproc{ref-brown_confirmatory_2015}{2015}): CFI \textgreater{} 0.95;
TLI \textgreater{} 0.95; RMSEA \textless{} 0.06.

Para examinar la estabilidad métrica del modelo de medición
(\citeproc{ref-davidov_measurement_2014}{Davidov et al., 2014}),
realizamos pruebas de invarianza longitudinal utilizando los datos de
las dos olas del estudio. En línea con Liu et al.
(\citeproc{ref-liu_testing_2017}{2017}), seguimos un enfoque jerárquico
de cuatro modelos: configural (estructura factorial equivalente), débil
(igualdad de cargas factoriales), fuerte (igualdad de interceptos) y
estricto (igualdad de varianzas de error). Este enfoque resulta
especialmente pertinente para indicadores categóricos ordenados, dado
que tratar escalas Likert de cuatro puntos como continuas puede
introducir sesgos en las estimaciones. Además del criterio basado en el
cambio en el valor del chi-cuadrado, adoptamos como indicadores de
invarianza el cambio en el CFI (\(\Delta \geq -0.010\)) y el RMSEA
(\(\Delta \geq 0.015\)), siguiendo las recomendaciones de Chen (2007).

Finalmente, para evaluar la validez externa del modelo de medición,
llevamos a cabo un análisis de regresión entre variables latentes y
observadas (\citeproc{ref-kline_principles_2023}{Kline, 2023}). Este
análisis permitió explorar la relación entre factores meritocráticos y
no meritocráticos con las preferencias por la justicia de mercado en
salud, pensiones y educación, conforme a la literatura existente
(\citeproc{ref-castillo_socialization_2024}{Castillo et al., 2024}).

\section{Resultados}\label{resultados}

Los resultados muestran que la escala presenta un adecuado ajuste al
modelo propuesto, con indicadores sólidos (CFI = 0.989, TLI = 0.979,
RMSEA = 0.046, χ²(df = 14) = 39.183), confirmando la validez de sus
dimensiones en población escolar. Los análisis revelan altas cargas
factoriales (\textgreater{} 0.6) para todos los ítems en sus respectivos
factores latentes.

Asimismo, las correlaciones entre factores son consistentes con las
observadas en población adulta
(\citeproc{ref-castillo_multidimensional_2023}{Castillo et al., 2023}):
la percepción de elementos meritocráticos se asocia negativamente con la
percepción de elementos no meritocráticos, mientras que esta última
muestra una relación positiva con la preferencia por la meritocracia.
Estos hallazgos sugieren que, incluso en una etapa temprana de
socialización, los estudiantes diferencian entre cómo perciben el
funcionamiento de elementos meritocráticos y no meritocráticos, y cómo
prefieren que estos operen en la sociedad. Resulta especialmente notable
que una mayor percepción de no meritocracia esté vinculada con una mayor
preferencia por la misma, destacando el papel central de este principio
moral en la formación de valores y actitudes durante la etapa escolar
(\citeproc{ref-batruch_belief_2022}{Batruch et al., 2022};
\citeproc{ref-darnon_where_2018}{Darnon et al., 2018};
\citeproc{ref-wiederkehr_belief_2015}{Wiederkehr et al., 2015}).

Los resultados del análisis de invarianza longitudinal no respaldan la
estabilidad métrica del modelo. En el nivel de invarianza débil, que
implica la restricción de las cargas factoriales entre las dos olas, los
indicadores sugieren falta de ajuste adecuado (\(\Delta\) CFI = -0.006,
\(\Delta\) RMSEA = 0.006). Al examinar el modelo factorial en la segunda
ola, se observa que, aunque los ítems se relacionan positivamente con
sus respectivos factores, las cargas factoriales muestran cambios en el
orden jerárquico. En particular, los ítems relacionados con la
percepción o preferencia de los contactos adquieren mayor peso en los
factores de no meritocracia, desplazando la preeminencia previa del ítem
sobre ``padres ricos''. Este cambio podría explicarse por un proceso de
socialización en el cual, con el tiempo, los estudiantes adquieren una
mayor comprensión sobre el funcionamiento de los elementos no
meritocráticos y su relevancia en las interacciones sociales.

Los resultados de los modelos de regresión indican que una mayor
percepción y preferencia por la meritocracia se asocian positivamente
con una mayor preferencia por la justicia de mercado, mientras que esta
relación no se observa de manera equivalente con los factores no
meritocráticos. En particular, los análisis muestran que una mayor
percepción de no meritocracia se relaciona con una menor preferencia por
la justicia de mercado, aunque, paradójicamente, una mayor preferencia
por elementos no meritocráticos se asocia con una mayor preferencia por
la justicia de mercado. Aunque este hallazgo contraviene nuestra
hipótesis inicial, resulta coherente bajo la premisa de que preferir que
la meritocracia opere a través de contactos o riqueza familiar podría
estar vinculado con una mayor aceptación de que el acceso a servicios
sociales básicos dependa de la capacidad de pago individual (Lindh,
2015; Castillo et al., 2014; Lindh \& McCall, 2020).

\section{Pre-registro}\label{pre-registro}

El pre-registro del estudio puede encontrarse en este
\href{https://osf.io/2uhrk}{link}.

\section{Referencias}\label{referencias}

\phantomsection\label{refs}
\begin{CSLReferences}{1}{0}
\bibitem[\citeproctext]{ref-batruch_belief_2022}
Batruch, A., Jetten, J., Van de Werfhorst, H., Darnon, C., \& Butera, F.
(2022). Belief in {School Meritocracy} and the {Legitimization} of
{Social} and {Income Inequality}. \emph{Social Psychological and
Personality Science}, 194855062211110.
\url{https://doi.org/10.1177/19485506221111017}

\bibitem[\citeproctext]{ref-brown_confirmatory_2015}
Brown, T. A. (2015). \emph{Confirmatory factor analysis for applied
research} (Second edition). New York London: The Guilford Press.

\bibitem[\citeproctext]{ref-castillo_multidimensional_2023}
Castillo, J. C., Iturra, J., Maldonado, L., Atria, J., \& Meneses, F.
(2023). A {Multidimensional Approach} for {Measuring Meritocratic
Beliefs}: {Advantages}, {Limitations} and {Alternatives} to the {ISSP
Social Inequality Survey}. \emph{International Journal of Sociology},
\emph{53}(6), 448--472.
\url{https://doi.org/10.1080/00207659.2023.2274712}

\bibitem[\citeproctext]{ref-castillo_socialization_2024}
Castillo, J. C., Salgado, M., Carrasco, K., \& Laffert, A. (2024). The
{Socialization} of {Meritocracy} and {Market Justice Preferences} at
{School}. \emph{Societies}, \emph{14}(11), 214.
\url{https://doi.org/10.3390/soc14110214}

\bibitem[\citeproctext]{ref-chancel_world_2022}
Chancel, L., Piketty, T., Saez, E., \& Zucman, G. (2022). World
inequality report 2022.
https://bibliotecadigital.ccb.org.co/handle/11520/27510.

\bibitem[\citeproctext]{ref-corvalan_mercado_2017}
Corvalán, J., Carrasco, A., \& García-Huidobro;J. E. (2017).
\emph{Mercado escolar: {Libertad}, diversidad y desigualdad}. Ediciones
UC.

\bibitem[\citeproctext]{ref-darnon_where_2018}
Darnon, C., Wiederkehr, V., Dompnier, B., \& Martinot, D. (2018).
{``{Where} there is a will, there is a way''}: {Belief} in school
meritocracy and the social-class achievement gap. \emph{British Journal
of Social Psychology}, \emph{57}(1), 250--262.
\url{https://doi.org/10.1111/bjso.12214}

\bibitem[\citeproctext]{ref-davidov_measurement_2014}
Davidov, E., Meuleman, B., Cieciuch, J., Schmidt, P., \& Billiet, J.
(2014). Measurement {Equivalence} in {Cross-National Research}.
\emph{Annual Review of Sociology}, \emph{40}(Volume 40, 2014), 55--75.
\url{https://doi.org/10.1146/annurev-soc-071913-043137}

\bibitem[\citeproctext]{ref-kline_principles_2023}
Kline, R. B. (2023). \emph{Principles and {Practice} of {Structural
Equation Modeling}}. Guilford Publications.

\bibitem[\citeproctext]{ref-lindh_public_2015}
Lindh, A. (2015). Public {Opinion} against {Markets}? {Attitudes}
towards {Market Distribution} of {Social Services} -- {A Comparison} of
17 {Countries}. \emph{Social Policy \& Administration}, \emph{49}(7),
887--910. \url{https://doi.org/10.1111/spol.12105}

\bibitem[\citeproctext]{ref-liu_testing_2017}
Liu, Y., Millsap, R. E., West, S. G., Tein, J.-Y., Tanaka, R., \& Grimm,
K. J. (2017). Testing measurement invariance in longitudinal data with
ordered-categorical measures. \emph{Psychological Methods},
\emph{22}(3), 486--506. \url{https://doi.org/10.1037/met0000075}

\bibitem[\citeproctext]{ref-wiederkehr_belief_2015}
Wiederkehr, V., Bonnot, V., Krauth-Gruber, S., \& Darnon, C. (2015).
Belief in school meritocracy as a system-justifying tool for low status
students. \emph{Frontiers in Psychology}, \emph{6}.

\bibitem[\citeproctext]{ref-young_rise_1958}
Young, M. (1958). \emph{The rise of the meritocracy}. New Brunswick,
N.J., U.S.A: Transaction Publishers.

\end{CSLReferences}

\textbf{Word count}: 1662



\end{document}
